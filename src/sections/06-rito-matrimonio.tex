
% Rito del Matrimonio
\section*{Rito del Matrimonio}

	\begin{dialoghi}
		\item[\sacerdote] Carissimi Irene e Davide, siete venuti nella casa del Signore, davanti al ministro della Chiesa e davanti alla comunità, perché la vostra decisione di unirvi in matrimonio riceva il sigillo dello Spirito Santo, sorgente dell'amore fedele e inesauribile. Ora Cristo vi rende partecipi dello stesso amore con cui egli ha amato la sua Chiesa, fino a dare se stesso per lei. Vi chiedo pertanto di esprimere le vostre intenzioni.
	\end{dialoghi}

\subsection*{Domande}

	\begin{dialoghi}
		\item[\sacerdote] Irene e Davide siete venuti a contrarre matrimonio, senza alcuna costrizione, in piena libertà e consapevoli del significato della vostra decisione?
		\item[\sposi] Sì.
		\item[\sacerdote] Siete disposti nella nuova via del matrimonio ad amarvi e onorarvi l'un l'altro per tutta la vita?
		\item[\sposi] Sì.
		\item[\sacerdote] Siete disposti ad accogliere con amore i figli che Dio vorrà donarvi e a educarli secondo la legge di Cristo e della sua Chiesa?
		\item[\sposi] Sì.
	\end{dialoghi}

\subsection*{Consenso}

	\begin{dialoghi}
		\item[\sacerdote] Alla presenza di Dio e davanti alla Chiesa qui riunita, datevi la mano destra ed esprimete il vostro consenso. Il Signore, inizio e compimento del vostro amore, sia con voi sempre.
		\item[\sposo] Io Davide accolgo te, Irene, come mia sposa. Con la grazia di Cristo prometto di esserti fedele sempre, nella gioia e nel dolore, nella salute e nella malattia, e di amarti e onorarti tutti i giorni della mia vita.
		\item[\sposa] Io Irene accolgo te, Davide, come mio sposo. Con la grazia di Cristo prometto di esserti fedele sempre, nella gioia e nel dolore, nella salute e nella malattia, e di amarti e onorarti tutti i giorni della mia vita.
		\item[\sacerdote] Il Dio di Abramo, il Dio di Isacco, il Dio di Giacobbe, il Dio che nel paradiso ha unito Adamo ed Eva confermi in Cristo il consenso che avete manifestato davanti alla Chiesa e vi sostenga con la sua benedizione. L'uomo non osi separare ciò che Dio unisce.
		\item[\assemblea] Amen.
	\end{dialoghi}

\subsection*{Benedizione e consegna degli anelli}

	\begin{dialoghi}
		\item[\sacerdote] Signore, benedici e santifica l'amore di questi sposi: l'anello che porteranno come simbolo di fedeltà li richiami continuamente al vicendevole amore. Per Cristo nostro Signore.
		\item[\assemblea] Amen.
		\item[\sposo] Irene, ricevi questo anello, segno del mio amore e della mia fedeltà. Nel nome del Padre, del Figlio e dello Spirito Santo.
		\item[\sposa] Davide, ricevi questo anello, segno del mio amore e della mia fedeltà. Nel nome del Padre, del Figlio e dello Spirito Santo.
	\end{dialoghi}

\subsection*{Benedizione degli sposi}

	\begin{dialoghi}
		\item[\sacerdote] Fratelli e sorelle, invochiamo con fiducia il Signore, perché effonda la sua grazia e la sua benedizione su questi sposi che celebrano in Cristo il loro matrimonio: egli che li ha uniti nel patto santo [per la comunione al corpo e al sangue di Cristo] li confermi nel reciproco amore.

		O Dio, con la tua onnipotenza hai creato dal nulla tutte le cose e nell'ordine primordiale dell'universo hai formato l'uomo e la donna a tua immagine, donandoli l'uno all'altro come sostegno inseparabile, perché siano non più due, ma una sola carne; così hai insegnato che non è mai lecito separare ciò che tu hai costituito in unità.

		O Dio, in un mistero così grande hai consacrato l'unione degli sposi e hai reso il patto coniugale sacramento di Cristo e della Chiesa.

		O Dio, in te la donna e l'uomo si uniscono, e la prima comunità umana, la famiglia, riceve in dono quella benedizione che nulla poté cancellare, né il peccato originale né le acque del diluvio.

		Guarda ora con bontà questi tuoi figli che, uniti nel vincolo del matrimonio, chiedono l'aiuto della tua benedizione: effondi su di loro la grazia dello Spirito Santo perché, con la forza del tuo amore diffuso nei loro cuori, rimangano fedeli al patto coniugale.

		In questa tua figlia Irene, dimori il dono dell'amore e della pace e sappia imitare le donne sante lodate dalla Scrittura. Davide, suo sposo, viva con lei in piena comunione, la riconosca partecipe dello stesso dono di grazia, la onori come uguale nella dignità, la ami sempre con quell'amore con il quale Cristo ha amato la sua Chiesa.

		Ti preghiamo, Signore, affinché questi tuoi figli rimangano uniti nella fede e nell'obbedienza dei tuoi comandamenti; fedeli a un solo amore siano esemplari per integrità di vita; sostenuti dalla forza del Vangelo diano a tutti buona testimonianza di Cristo.

		Sia feconda la loro unione, diventino genitori saggi e forti e insieme possano vedere i figli dei loro figli. E dopo una vita lunga e serena giungano alla beatitudine eterna del regno dei cieli. Per Cristo nostro Signore.
		\item[\assemblea] Amen.
		\item[\sacerdote] Benediciamo il Signore.
		\item[\assemblea] A lui onore e gloria nei secoli.
	\end{dialoghi}

