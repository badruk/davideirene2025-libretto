\newpage
\subsection*{Lettura degli articoli del codice civile}

	\begin{dialoghi}
		\item[\sacerdote] Carissimi Irene e Davide, avete celebrato il sacramento del Matrimonio manifestando il vostro consenso dinanzi a me ed ai testimoni. Oltre la grazia divina e gli effetti stabiliti dai sacri Canoni, il vostro Matrimonio produce anche gli effetti civili secondo le leggi dello Stato. Vi do quindi lettura degli articoli del Codice civile riguardanti i diritti e i doveri dei coniugi che voi siete tenuti a rispettare ed osservare:

		\textbf{Art. 143:} Con il matrimonio il marito e la moglie acquistano gli stessi diritti e assumono i medesimi doveri. Dal matrimonio deriva l'obbligo reciproco alla fedeltà, all'assistenza morale e materiale, alla collaborazione nell'interesse della famiglia e alla coabitazione.

		Entrambi i coniugi sono tenuti, ciascuno in relazione alle proprie sostanze e alla propria capacità di lavoro professionale o casalingo, a contribuire ai bisogni della famiglia.

		\textbf{Art. 144:} I coniugi concordano tra loro l'indirizzo della vita familiare e fissano la residenza della famiglia secondo le esigenze di entrambi e quelle preminenti della famiglia stessa. A ciascuno dei coniugi spetta il potere di attuare l'indirizzo concordato.

		\textbf{Art. 147:} «Il matrimonio impone ad ambedue i coniugi l'obbligo di mantenere, istruire, educare e assistere moralmente i figli, nel rispetto delle loro capacità, inclinazioni naturali e aspirazioni, secondo quanto previsto dall'articolo 315-bis».

		L'art. 315-bis del codice civile (Diritti e doveri del figlio) così dispone:

		«Il figlio ha diritto di essere mantenuto, educato, istruito e assistito moralmente dai genitori, nel rispetto delle sue capacità, delle sue inclinazioni naturali e delle sue aspirazioni. Il figlio ha diritto di crescere in famiglia e di mantenere rapporti significativi con i parenti. Il figlio minore che abbia compiuto gli anni dodici, e anche di età inferiore ove capace di discernimento, ha diritto di essere ascoltato in tutte le questioni e le procedure che lo riguardano. Il figlio deve rispettare i genitori e deve contribuire, in relazione alle proprie capacità, alle proprie sostanze e al proprio reddito, al mantenimento della famiglia finché convive con essa».
	\end{dialoghi}

\subsection*{Firme}

	\begin{dialoghi}
		\item[\sacerdote] (Spazio per le firme degli sposi e dei testimoni)
	\end{dialoghi}