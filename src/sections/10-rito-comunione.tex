

\section*{Riti di Comunione}

	\begin{dialoghi}
		\item[\sacerdote] Obbedienti alla parola del Salvatore e formati al suo divino insegnamento, osiamo dire:
		\item[\assemblea] Padre nostro, che sei nei cieli, sia santificato il tuo nome, venga il tuo regno, sia fatta la tua volontà, come in cielo così in terra. Dacci oggi il nostro pane quotidiano, e rimetti a noi i nostri debiti come anche noi li rimettiamo ai nostri debitori, e non abbandonarci alla tentazione, ma liberaci dal male.
		\item[\sacerdote] Liberaci, o Signore, da tutti i mali, concedi la pace ai nostri giorni; e con l'aiuto della tua misericordia, vivremo sempre liberi dal peccato e sicuri da ogni turbamento, nell'attesa che si compia la beata speranza, e venga il nostro Salvatore Gesù Cristo.
		\item[\assemblea] Tuo è il regno, tua la potenza e la gloria nei secoli.
		\item[\sacerdote] Signore Gesù Cristo, che hai detto ai tuoi apostoli: \textquotedblleft Vi lascio la pace, vi do la mia pace\textquotedblright, non guardare ai nostri peccati, ma alla fede della tua Chiesa, e donale unità e pace secondo la tua volontà. Tu che vivi e regni nei secoli dei secoli.
		\item[\assemblea] Amen.
		\item[\sacerdote] La pace del Signore sia sempre con voi.
		\item[\assemblea] E con il tuo spirito.
		\item[\sacerdote] In Cristo, che ci ha resi tutti fratelli con la sua croce, scambiatevi un segno di riconciliazione e di pace.

		\item[\assemblea] Agnello di Dio, che togli i peccati del mondo, abbi pietà di noi. Agnello di Dio, che togli i peccati del mondo, abbi pietà di noi. Agnello di Dio, che togli i peccati del mondo, dona a noi la pace.
		\item[\sacerdote] Ecco l'Agnello di Dio, ecco colui che toglie i peccati del mondo. Beati gli invitati alla cena dell'Agnello.
		\item[\assemblea] O Signore, non son degno di partecipare alla tua mensa: ma dì soltanto una parola e io sarò salvato.
	\end{dialoghi}

\subsection*{Canto di Comunione: \textit{Il canto dell'amore}}

	\begin{mystrofe}
		Se dovrai attraversare il deserto \\
		Non temere io sarò con te \\
		Se dovrai camminare nel fuoco \\
		La sua fiamma non ti brucerà \\
		Seguirai la mia luce nella notte \\
		Sentirai la mia forza nel cammino \\
		Io sono il tuo Dio, il Signore \\
	\end{mystrofe}

	\hfill

	\begin{mystrofe}
		Sono io che ti ho fatto e plasmato \\
		Ti ho chiamato per nome \\
		Io da sempre ti ho conosciuto \\
		E ti ho dato il mio amore \\
		Perché tu sei prezioso ai miei occhi \\
		Vali più del più grande dei tesori \\
		Io sarò con te dovunque andrai
	\end{mystrofe}

	\hfill

	\begin{mystrofe}
		Non pensare alle cose di ieri \\
		Cose nuove fioriscono già \\
		Aprirò nel deserto sentieri \\
		Darò acqua nell'aridità \\
	\end{mystrofe}
		
	\begin{mystrofe}
		\ritornello{Perché tu sei prezioso ai miei occhi \\
		Vali più del più grande dei tesori \\
		Io sarò con te dovunque andrai \\}
	\end{mystrofe}

	\hfill

	\begin{mystrofe}
		\ritornello{Perché tu sei prezioso ai miei occhi \\
		Vali più del più grande dei tesori \\
		Io sarò con te dovunque andrai \\}
	\end{mystrofe}

	\begin{mystrofe}
		Io ti sarò accanto sarò con te \\
		Per tutto il tuo viaggio sarò con te
	\end{mystrofe}

\subsection*{Canto di ringraziamento: \textit{Stai con me}}

	\begin{mystrofe}
		Stai con me, proteggimi \\
		Coprimi con le tue ali, o Dio \\
	\end{mystrofe}

	\begin{mystrofe}
		\ritornello{Quando la tempesta arriverà \\
		Volerò più in alto insieme a te \\
		Nelle avversità sarai con me \\
		Ed io saprò che tu sei il mio Re.}
	\end{mystrofe}

	\begin{mystrofe}
	Il cuore mio riposa in te \\
	Io vivrò in pace e verità \\
	\end{mystrofe}

	\begin{mystrofe}
		\ritornello{Quando la tempesta arriverà \\
		Volerò più in alto insieme a te \\
		Nelle avversità sarai con me \\
		Ed io saprò che tu sei il mio Re.}
	\end{mystrofe}

\subsection*{Orazione dopo la comunione}

\begin{dialoghi}
	\item[\sacerdote] Preghiamo. O Padre, che ci hai accolti alla tua mensa, concedi a questa nuova famiglia, consacrata dalla tua benedizione, di essere sempre fedele a te e di testimoniare il tuo amore nella comunità dei fratelli. Per Cristo nostro Signore.
	\item[\assemblea] Amen.
\end{dialoghi}
