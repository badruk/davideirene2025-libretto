
% Liturgia della Parola
\section*{LITURGIA DELLA PAROLA}

\subsection*{PRIMA LETTURA}

\begin{dialoghi}
\item[Lettore] \textbf{Dal libro della Sapienza} (Sap 9, 13-18)

Quale uomo può conoscere il volere di Dio?

Chi può immaginare che cosa vuole il Signore?

I ragionamenti dei mortali sono timidi

e incerte le nostre riflessioni,

perché un corpo corruttibile appesantisce l'anima

e la tenda d'argilla opprime una mente piena di preoccupazioni.

A stento immaginiamo le cose della terra,

scopriamo con fatica quelle a portata di mano;

ma chi ha investigato le cose del cielo?

Chi avrebbe conosciuto il tuo volere,

se tu non gli avessi dato la sapienza

e dall'alto non gli avessi inviato il tuo santo spirito?

Così vennero raddrizzati i sentieri di chi è sulla terra;

gli uomini furono istruiti in ciò che ti è gradito

e furono salvati per mezzo della sapienza.

Parola di Dio.
\item[Assemblea] Rendiamo grazie a Dio.
\end{dialoghi}

\subsection*{SALMO RESPONSORIALE} (Sal. 89)

\begin{dialoghi}
\item[Assemblea] \textit{Signore, sei stato per noi un rifugio di generazione in generazione.}
\item[Lettore] Tu fai ritornare l'uomo in polvere,

quando dici: "Ritornate, figli dell'uomo".

Mille anni, ai tuoi occhi,

sono come il giorno di ieri che è passato,

come un turno di veglia nella notte.
\item[Assemblea] \textit{Signore, sei stato per noi un rifugio di generazione in generazione.}
\item[Lettore] Tu li sommergi:

sono come un sogno al mattino,

come l'erba che germoglia;

al mattino fiorisce e germoglia,

alla sera è falciata e secca.
\item[Assemblea] \textit{Signore, sei stato per noi un rifugio di generazione in generazione.}
\item[Lettore] Insegnaci a contare i nostri giorni

E acquisteremo un cuore saggio.

Ritorna, Signore: fino a quando?

Abbi pietà dei tuoi servi!
\item[Assemblea] \textit{Signore, sei stato per noi un rifugio di generazione in generazione.}
\item[Lettore] Saziaci al mattino con il tuo amore:

esulteremo e gioiremo per tutti i nostri giorni.

Sia su di noi la dolcezza del Signore, nostro Dio:

rendi salda per noi l'opera delle nostre mani,

l'opera delle nostre mani rendi salda.
\item[Assemblea] \textit{Signore, sei stato per noi un rifugio di generazione in generazione.}
\end{dialoghi}

\subsection*{SECONDA LETTURA}

\begin{dialoghi}
\item[Lettore] \textbf{Dalla prima lettera di san Paolo apostolo ai Corinzi} (1Cor 12,31b-14,1a)

Fratelli, vi mostrerò una via migliore di tutte.

Se anche parlassi le lingue degli uomini e degli angeli, ma non avessi la carità, sono come un bronzo che risuona o un cembalo che tintinna.

E se avessi il dono della profezia e conoscessi tutti i misteri e tutta la scienza, e possedessi la pienezza della fede così da trasportare le montagne, ma non avessi la carità, non sono nulla.

E se anche distribuissi tutte le mie sostanze e dessi il mio corpo per esser bruciato, ma non avessi la carità, niente mi giova.

La carità è paziente, è benigna la carità; non è invidiosa la carità, non si vanta, non si gonfia, non manca di rispetto, non cerca il suo interesse, non si adira, non tiene conto del male ricevuto, non gode dell'ingiustizia, ma si compiace della verità. Tutto copre, tutto crede, tutto spera, tutto sopporta.

La carità non avrà mai fine. Le profezie scompariranno; il dono delle lingue cesserà e la scienza svanirà. La nostra conoscenza è imperfetta e imperfetta la nostra profezia. Ma quando verrà ciò che è perfetto, quello che è imperfetto scomparirà. Quand'ero bambino, parlavo da bambino, pensavo da bambino, ragionavo da bambino. Ma, divenuto uomo, ciò che era da bambino l'ho abbandonato.

Ora vediamo come in uno specchio, in maniera confusa; ma allora vedremo a faccia a faccia. Ora conosco in modo imperfetto, ma allora conoscerò perfettamente, come anch'io sono conosciuto. Queste dunque le tre cose che rimangono: la fede, la speranza e la carità; ma di tutte più grande è la carità!

Ricercate la carità.

Parola di Dio.
\item[Assemblea] Rendiamo grazie a Dio.
\end{dialoghi}

\subsection*{CANTO AL VANGELO: \textit{Alleluia}}

\begin{verse}
\begin{mystrofe}
RIT. Alleluia...
\end{mystrofe}

\begin{mystrofe}
Canto per Cristo che mi libererà quando verrà nella gloria, quando la vita con lui rinascerà, alleluia, alleluia!
\end{mystrofe}

\ritornello{RIT.}
\end{verse}

\subsection*{VANGELO}

\begin{dialoghi}
\item[Lettore] \textbf{Dal Vangelo secondo Luca} (Lc 14, 25-33)

In quel tempo, una folla numerosa andava con Gesù. Egli si voltò e disse loro: "Se uno viene a me e non mi ama più di quanto ami suo padre, la madre, la moglie, i figli, i fratelli, le sorelle e perfino la propria vita, non può essere mio discepolo. Colui che non porta la propria croce e non viene dietro a me, non può essere mio discepolo. Chi di voi, volendo costruire una torre, non siede prima a calcolare la spesa e a vedere se ha i mezzi per portarla a termine? Per evitare che, se getta le fondamenta e non è in grado di finire il lavoro, tutti coloro che vedono comincino a deriderlo, dicendo: "Costui ha iniziato a costruire, ma non è stato capace di finire il lavoro". Oppure quale re, partendo in guerra contro un altro re, non siede prima a esaminare se può affrontare con diecimila uomini chi gli viene incontro con ventimila? Se no, mentre l'altro è ancora lontano, gli manda dei messaggeri per chiedere pace.

Così chiunque di voi non rinuncia a tutti i suoi averi, non può essere mio discepolo".

Parola del Signore.
\item[Assemblea] Lode a Te o Cristo.
\end{dialoghi}

\subsection*{OMELIA DEL SACERDOTE}
