
% Liturgia Eucaristica
\section*{Liturgia eucaristica}

\subsection*{Canto di offertorio: \textit{Benedici}}

		\begin{mystrofe}
			Nebbia e freddo, giorni lunghi e amari \\
			mentre il seme muore. \\
			Poi il prodigio antico e sempre nuovo \\
			del primo filo d'erba \\
			e nel vento dell'estate ondeggiano le spighe: \\
			avremo ancora pane.
		\end{mystrofe}
		
		\begin{mystrofe}
			\ritornello{Benedici, o Signore, \\
			questa offerta che portiamo a Te. \\
			Facci uno come il pane \\
			che anche oggi hai dato a noi.}
		\end{mystrofe}

		\begin{mystrofe}
			Nei filari, dopo il lungo inverno fremono le viti. \\
			La rugiada avvolge nel silenzio i primi tralci verdi, \\
			poi i colori dell'autunno coi grappoli maturi: \\
			avremo ancora vino.
		\end{mystrofe}

		\begin{mystrofe}
			\ritornello{Benedici, o Signore, \\
			questa offerta che portiamo a Te. \\
			Facci uno come il vino \\
			che anche oggi hai dato a noi.}
		\end{mystrofe}

\subsection*{Presentazione dei Doni}

	\begin{dialoghi}
		\item[\sacerdote] Benedetto sei tu, Signore, Dio dell'universo: dalla tua bontà abbiamo ricevuto questo pane, frutto della terra e del lavoro dell'uomo; lo presentiamo a te perché diventi per noi cibo di vita eterna.
		\item[\assemblea] Benedetto nei secoli il Signore
		\item[\sacerdote] Benedetto sei tu, Signore, Dio dell'universo: dalla tua bontà abbiamo ricevuto questo vino, frutto della vite e del lavoro dell'uomo; lo presentiamo a te perché diventi per noi bevanda di salvezza.
		\item[\assemblea] Benedetto nei secoli il Signore
		\item[\sacerdote] Pregate, fratelli, perché il mio e vostro sacrificio sia gradito a Dio, Padre onnipotente.
		\item[\assemblea] Il Signore riceva dalle tue mani questo sacrificio a lode e gloria del suo nome, per il bene nostro e di tutta la sua santa Chiesa.
	\end{dialoghi}

\subsection*{Orazione sui Doni}

	\begin{dialoghi}
		\item[\sacerdote] O Dio, Padre di bontà, accogli il pane e il vino, che la tua famiglia ti offre con intima gioia, e custodisci nel tuo amore Irene e Davide che hai unito col sacramento nuziale. Per Cristo nostro Signore.
		\item[\assemblea] Amen
	\end{dialoghi}

\subsection*{Preghiera Eucaristica}

	\begin{dialoghi}
		\item[\sacerdote] Il Signore sia con voi
		\item[\assemblea] E con il tuo spirito
		\item[\sacerdote] In alto i nostri cuori
		\item[\assemblea] Sono rivolti al Signore
		\item[\sacerdote] Rendiamo grazie al Signore, nostro Dio
		\item[\assemblea] È cosa buona e giusta
		\item[\sacerdote] È veramente cosa buona e giusta, nostro dovere e fonte di salvezza, rendere grazie sempre, qui e in ogni luogo, a te, Signore, Padre santo, Dio onnipotente ed eterno. Tu hai dato alla comunità coniugale la dolce legge dell'amore e il vincolo indissolubile della pace perché l'unione casta e feconda degli sposi accresca il numero dei tuoi figli. Con disegno mirabile hai disposto che la nascita di nuove creature allieti l'umana famiglia e la loro rinascita in Cristo edifichi la tua Chiesa. Per questo mistero di salvezza, uniti agli angeli e a tutti i santi, cantiamo insieme l'inno della tua lode.
	\end{dialoghi}

\subsection*{Canto: \textit{Santo (Jesus Christ Superstar)}}

	\begin{mystrofe}
		Santo, santo, santo è il Signore \\
		il Signore Dio dell'Universo. \\
		I cieli e la terra sono pieni della tua gloria \\
		osanna nell'alto dei cieli. \\
		Benedetto sia colui che viene, \\
		che viene nel nome del Signore. \\
		Osanna, osanna nell'alto dei cieli \\
		osanna nell'alto dei cieli.
	\end{mystrofe}

	\begin{dialoghi}
		\item[\sacerdote] Veramente santo sei tu, o Padre, fonte di ogni santità. Ti preghiamo: santifica questi doni con la rugiada del tuo Spirito perché diventino per noi il corpo e (+) il sangue del Signore nostro Gesù Cristo.

		Egli, consegnandosi volontariamente alla passione, prese il pane, rese grazie, lo spezzò, lo diede ai suoi discepoli, e disse: "\textbf{PRENDETE, E MANGIATENE TUTTI: QUESTO É IL MIO CORPO OFFERTO IN SACRIFICIO PER VOI.}"

		Allo stesso modo, dopo aver cenato, prese il calice, di nuovo ti rese grazie, lo diede ai suoi discepoli e disse: "\textbf{PRENDETE, E BEVETENE TUTTI: QUESTO É IL CALICE DEL MIO SANGUE PER LA NUOVA ED ETERNA ALLEANZA, VERSATO PER VOI E PER TUTTI IN REMISSIONE DEI PECCATI. FATE QUESTO IN MEMORIA DI ME.}"

		Mistero della fede.
		\item[\assemblea] Annunciamo la tua morte, Signore, proclamiamo la tua risurrezione, nell'attesa della tua venuta.
		\item[\sacerdote] Celebrando il memoriale della morte e risurrezione del tuo Figlio, ti offriamo, Padre, il pane della vita e il calice della salvezza, e ti rendiamo grazie perché ci hai resi degni di stare alla tua presenza a compiere il servizio sacerdotale. Ti preghiamo umilmente: per la comunione al corpo e al sangue di Cristo lo Spirito Santo ci riunisca in un solo corpo.

		Ricordati, Padre, della tua Chiesa diffusa su tutta la terra e qui convocata nel giorno in cui Cristo ha vinto la morte e ci ha resi partecipi della sua vita immortale: rendila perfetta nell'amore in unione con il nostro Papa Leone, il nostro Vescovo Mario, i presbiteri e i diaconi.

		Ricordati dei tuoi figli Irene e Davide, che in Cristo hanno costituito una nuova famiglia, piccola Chiesa e sacramento del tuo amore, perché la grazia di questo giorno si estenda a tutta la loro vita.

		Ricordati anche dei nostri fratelli e sorelle, che si sono addormentati nella speranza della risurrezione e, nella tua misericordia, di tutti i defunti: ammettili a godere la luce del tuo volto.

		Di noi tutti abbi misericordia, donaci di aver parte alla vita eterna, insieme con la beata Maria, Vergine e Madre di Dio, San Giuseppe suo sposo, gli apostoli e tutti i santi, che in ogni tempo ti furono graditi: e in Gesù Cristo tuo Figlio canteremo la tua gloria.

		Per Cristo, con Cristo e in Cristo, a te, Dio, Padre onnipotente, nell'unita dello Spirito Santo, ogni onore e gloria, per tutti i secoli dei secoli.
		\item[\assemblea] Amen.
	\end{dialoghi}