% Inserire logo prima pagina
% Amen professione di Fede
% Pagina 1 all'inizio del primo canto
% 1 cor 13


\documentclass[11pt, a5paper]{article}
\usepackage[utf8]{inputenc}
\usepackage[T1]{fontenc}
\usepackage[italian]{babel}
\usepackage[a5paper, twoside,
  inner=1.5cm, % margine interno (vicino alla piega)
  outer=1.5cm, % margine esterno
  top=1.5cm,
  bottom=1.5cm,
  bindingoffset=0.5cm % spazio extra per la rilegatura
]{geometry}

\usepackage{fontspec}
\usepackage{xcolor}
\usepackage{lipsum}
\usepackage{verse}
\usepackage{multicol}
\usepackage{enumitem}
\usepackage{graphicx}
\usepackage{fancyhdr}
\usepackage{titlesec}



% Sezione più piccola
\titleformat{\section}
  {\centering\normalfont\normalsize\bfseries} % Stile: grandezza, font, grassetto
  {\thesection}{1em}{}         % Numerazione, spaziatura

% Sottosezione più piccola
\titleformat{\subsection}
  {\centering\normalfont\normalsize\bfseries}
  {\thesubsection}{1em}{}

% Font Brittany
\newfontfamily{\brittany}{BrittanySignature}[
  Path = assets/fonts/,
  Extension = .ttf,
  UprightFont = *
]

% Comandi per dialoghi
\newcommand{\tutti}{\underline{Tutti:}\quad}
\newcommand{\sacerdote}{\underline{Sacerdote:}\quad}
\newcommand{\sacerdotetutti}{\underline{Sacerdote - Tutti:}\quad}
\newcommand{\assemblea}{\underline{Tutti:}\quad}
\newcommand{\sposi}{\underline{Sposi:}\quad}
\newcommand{\lettore}{\underline{Lettore:}\quad}
\newcommand{\sposo}{\underline{Davide:}\quad}
\newcommand{\sposa}{\underline{Irene:}\quad}

% Ambiente per dialoghi
\newlist{dialoghi}{description}{1}
\setlist[dialoghi]{
  font=\normalfont,
  labelwidth=2cm,
  leftmargin=2.25cm,
  style=nextline
}

% Ambiente per strofe
\newenvironment{mystrofe}{\noindent\centering \itshape \par}


\newcommand{\ritornello}[1]{%
  \par\noindent % Assicura che non ci siano rientri indesiderati
  \begin{center}
    RIT: \itshape #1%
  \end{center}%
  \par % Chiude correttamente il paragrafo
}


\newcommand{\professione}[1]{%
  \par\noindent % Assicura che non ci siano rientri indesiderati
    #1%
  \par % Chiude correttamente il paragrafo
}

% Intestazione e piè di pagina
\pagestyle{fancy}
\fancyhf{}
\renewcommand{\headrulewidth}{0pt}
\fancyfoot[C]{\thepage}

\newcommand{\lettura}[3]{
  \begin{center}
    \textbf{\textsc{#1}}{ (\textsc{#2})}\\

    \vspace{0.2cm}

    \begin{minipage}{0.9\textwidth}
      #3
    \end{minipage}
  
  \end{center}
  

  \begin{flushright}
    Parola di Dio
  \end{flushright}
  
  \begin{dialoghi}
    \item[\underline{Tutti:}] Rendiamo grazie a Dio.
  \end{dialoghi}
}

\newcommand{\vangelo}[3]{
  \begin{center}
    \textbf{\textsc{#1}}{ (\textsc{#2})}\\

    \vspace{0.2cm}

    \begin{minipage}{0.9\textwidth}
      #3
    \end{minipage}
  
  \end{center}
  

  \begin{flushright}
    Parola di Dio
  \end{flushright}
  
  \begin{dialoghi}
    \item[\underline{Tutti:}] Rendiamo grazie a Dio.
  \end{dialoghi}
}

% Font principale: Garamond
\setmainfont{EB Garamond Medium}[
  Ligatures=TeX,
  Numbers=OldStyle
]

\begin{document}

\newgeometry{
  top=1.25cm,
  bottom=1.25cm,
  left=1cm,
  right=1cm
}


  % Titolo
  \begin{center}
    \color{mymarriage}
    {\Large\textsc{Celebrazione del Sacramento del Matrimonio}}

    \vfill

    {\brittany\fontsize{65}{75}\selectfont Davide \& Irene}

    \vspace{0.5cm}

    % \includegraphics[width=4.25in, height=4.25in]{image1}

    \vfill

    {\Large \textsc{7 Settembre 2025}}
    \vspace{0.5cm}

    \textsc{\Large Parrocchia di San Pio X - Massa}
  \end{center}
  
  \thispagestyle{empty}

\newpage
\newpage
\thispagestyle{empty}
\mbox{}   % pagina bianca 1

\restoregeometry % ritorna alle impostazioni originali

% --------------------------------------------------
\newpage
\thispagestyle{empty}
\mbox{}   % pagina bianca 1

\newpage

\setcounter{page}{1}

% Canto iniziale 
\section*{Canto iniziale: \textit{Chiamati per nome}}

	\begin{mystrofe}
		\ritornello{Veniamo da te \\
			Chiamati per nome \\
			Che festa, Signore, tu cammini con noi \\
			Ci parli di te \\
			Per noi spezzi il pane \\
			Ti riconosciamo e il cuore arde, sei tu \\
			E noi tuo popolo \\
			Siamo qui}
	\end{mystrofe}

	\begin{mystrofe}
		Siamo come terra ed argilla \\
		E la tua parola ci plasmerà \\
		Brace pronta per la scintilla \\
		E il tuo spirito soffierà \\
		C'infiammerà
	\end{mystrofe}

	\begin{mystrofe}
		\ritornello{}
	\end{mystrofe}

	\begin{mystrofe}
		Siamo come semi nel solco \\
		Come vigna che il suo frutto darà \\
		Grano del Signore risorto \\
		La tua messe che fiorirà d'eternità
	\end{mystrofe}

	\begin{mystrofe}
		\ritornello{}
	\end{mystrofe}





% --------------------------------------------------

% Riti di introduzione
\section*{Rito di introduzione}

\begin{dialoghi}
	\item[Sacerdote] Nel nome del Padre e del Figlio e dello Spirito Santo
	\item[Tutti] Amen
	\item[Sacerdote] Il Signore che guida i nostri cuori nell'amore e nella pazienza di Cristo sia con tutti voi.
	\item[Tutti] E con il tuo spirito
\end{dialoghi}



% --------------------------------------------------

\subsection*{MEMORIA DEL BATTESIMO}

\begin{dialoghi}
\item[Sacerdote] Fratelli ci siamo riuniti con gioia nella casa del Signore nel giorno in cui Irene e Davide intendono formare la loro famiglia. In quest'ora di particolare grazia siamo loro vicini con l'affetto, con l'amicizia e la preghiera fraterna. Ascoltiamo insieme con loro la Parola che Dio oggi ci rivolge. In unione con la santa Chiesa supplichiamo Dio Padre, per Cristo Signore nostro, perché benedica questi suoi figli che stanno per celebrare il loro matrimonio, li accolga nel suo amore e li costituisca in unità. Facciamo ora memoria del Battesimo, nel quale siamo rinati a vita nuova. Divenuti figli nel Figlio, riconosciamo con gratitudine il dono ricevuto, per rimanere fedeli all'amore a cui siamo stati chiamati.
\item[Sacerdote] Padre, nel Battesimo del tuo Figlio Gesù al fiume Giordano hai rivelato al mondo l'amore sponsale per il tuo popolo.
\item[Assemblea] Noi ti lodiamo e ti rendiamo grazie.
\item[Sacerdote] Cristo Gesù, dal tuo costato aperto sulla Croce hai generato la Chiesa, tua diletta sposa.
\item[Assemblea] Noi ti lodiamo e ti rendiamo grazie.
\item[Sacerdote] Spirito Santo, potenza del Padre e del Figlio, oggi fai risplendere in Irene e Davide la veste nuziale della Chiesa.
\item[Assemblea] Noi ti lodiamo e ti rendiamo grazie.
\item[Sacerdote] Dio onnipotente, origine e fonte della vita, che ci hai rigenerati nell'acqua con la potenza del tuo Spirito, ravviva in noi la grazia del Battesimo, e concedi a Irene e Davide un cuore libero e una fede ardente perché, purificati nell'intimo, accolgano il dono del Matrimonio, nuova via della loro santificazione. Per Cristo nostro Signore.
\item[Assemblea] Amen
\end{dialoghi}


% --------------------------------------------------

\subsection*{GLORIA A DIO}

\begin{dialoghi}
\item[Assemblea] \textbf{Gloria a Dio nell'alto dei cieli e pace in terra agli uomini amati dal Signore. Noi ti lodiamo, ti benediciamo, ti adoriamo, ti glorifichiamo, ti rendiamo grazie per la tua gloria immensa, Signore Dio, re del cielo, Dio Padre onnipotente. Signore, Figlio unigenito, Gesù Cristo, Signore Dio, Agnello di Dio, Figlio del Padre, tu che togli i peccati del mondo, abbi pietà di noi; tu che togli i peccati del mondo accogli la nostra supplica; tu che siedi alla destra del Padre, abbi pietà di noi. Perché tu solo il santo, tu solo il Signore, tu solo l'Altissimo, Gesù Cristo, con lo Spirito Santo nella gloria di Dio Padre. Amen.}
\item[Sacerdote] Preghiamo

Ascolta, Signore, la nostra preghiera ed effondi con bontà la tua grazia su Irene e Davide, perché, unendosi davanti al tuo altare, siano confermati nel reciproco amore. Per il nostro Signore Gesù Cristo, tuo Figlio, che è Dio, e vive e regna con te nell'unità dello Spirito Santo, per tutti i secoli dei secoli.
\item[Assemblea] Amen.
\end{dialoghi}


% --------------------------------------------------

% Liturgia della Parola
\section*{Liturgia della Parola}

\subsection*{Prima lettura}

	\lettura{Dal libro della Sapienza}{Sap 9, 13-18}{
	Quale uomo può conoscere il volere di Dio?\\
	Chi può immaginare che cosa vuole il Signore?\\
	I ragionamenti dei mortali sono timidi \\
	e incerte le nostre riflessioni,\\
	perché un corpo corruttibile appesantisce l'anima\\
	e la tenda d'argilla opprime una mente piena di preoccupazioni.\\
	A stento immaginiamo le cose della terra,\\
	scopriamo con fatica quelle a portata di mano;\\
	ma chi ha investigato le cose del cielo?\\
	Chi avrebbe conosciuto il tuo volere,\\
	se tu non gli avessi dato la sapienza\\
	e dall'alto non gli avessi inviato il tuo santo spirito?\\
	Così vennero raddrizzati i sentieri di chi è sulla terra;\\
	gli uomini furono istruiti in ciò che ti è gradito\\
	e furono salvati per mezzo della sapienza.}

\subsection*{Salmo Responsoriale} (Sal. 89)

\begin{dialoghi}
	\item[\assemblea] \textit{Signore, sei stato per noi un rifugio di generazione in generazione.}
	\item[\lettore] Tu fai ritornare l'uomo in polvere,
	quando dici: "Ritornate, figli dell'uomo".
	Mille anni, ai tuoi occhi,
	sono come il giorno di ieri che è passato,
	come un turno di veglia nella notte.
	\item[\assemblea] \textit{Signore, sei stato per noi un rifugio di generazione in generazione.}
	\item[\lettore] Tu li sommergi:
	sono come un sogno al mattino,
	come l'erba che germoglia;
	al mattino fiorisce e germoglia,
	alla sera è falciata e secca.
	\item[\assemblea] \textit{Signore, sei stato per noi un rifugio di generazione in generazione.}
	\item[\lettore] Insegnaci a contare i nostri giorni
	E acquisteremo un cuore saggio.
	Ritorna, Signore: fino a quando?
	Abbi pietà dei tuoi servi!
	\item[\assemblea] \textit{Signore, sei stato per noi un rifugio di generazione in generazione.}
	\item[\lettore] Saziaci al mattino con il tuo amore:
	esulteremo e gioiremo per tutti i nostri giorni.
	Sia su di noi la dolcezza del Signore, nostro Dio:
	rendi salda per noi l'opera delle nostre mani,
	l'opera delle nostre mani rendi salda.
	\item[\assemblea] \textit{Signore, sei stato per noi un rifugio di generazione in generazione.}
\end{dialoghi}

\subsection*{Seconda lettura}

	\lettura{Dalla prima lettera di San Paolo Apostolo ai Corinzi}{1Cor 13}{
	Se parlassi le lingue degli uomini e degli angeli, ma non avessi la carità, sarei come bronzo che rimbomba o come cimbalo che strepita.\\
	E se avessi il dono della profezia, se conoscessi tutti i misteri e avessi tutta la conoscenza, se possedessi tanta fede da trasportare le montagne, ma non avessi la carità, non sarei nulla. \\
	E se anche dessi in cibo tutti i miei beni e consegnassi il mio corpo per averne vanto, ma non avessi la carità, a nulla mi servirebbe. \\
	La carità è magnanima, benevola è la carità; non è invidiosa, non si vanta, non si gonfia d'orgoglio, \\
	non manca di rispetto, non cerca il proprio interesse, non si adira, non tiene conto del male ricevuto, \\
	non gode dell'ingiustizia ma si rallegra della verità. \\
	Tutto scusa, tutto crede, tutto spera, tutto sopporta. \\
	La carità non avrà mai fine. Le profezie scompariranno, il dono delle lingue cesserà e la conoscenza svanirà. \\
	Infatti, in modo imperfetto noi conosciamo e in modo imperfetto profetizziamo. \\
	Ma quando verrà ciò che è perfetto, quello che è imperfetto scomparirà. \\
	Quand'ero bambino, parlavo da bambino, pensavo da bambino, ragionavo da bambino. Divenuto uomo, ho eliminato ciò che è da bambino.\\
	Adesso noi vediamo in modo confuso, come in uno specchio; allora invece vedremo faccia a faccia. Adesso conosco in modo imperfetto, ma allora conoscerò perfettamente, come anch'io sono conosciuto. \\
	Ora dunque rimangono queste tre cose: la fede, la speranza e la carità. Ma la più grande di tutte è la carità!}




	% \lettura{Dalla prima lettera di San Paolo Apostolo ai Corinzi}{1Cor 12,31b-14,1a}{
	% Fratelli, vi mostrerò una via migliore di tutte.
	% Se anche parlassi le lingue degli uomini e degli angeli, ma non avessi la carità, sono come un bronzo che risuona o un cembalo che tintinna.
	% E se avessi il dono della profezia e conoscessi tutti i misteri e tutta la scienza, e possedessi la pienezza della fede così da trasportare le montagne, ma non avessi la carità, non sono nulla.
	% E se anche distribuissi tutte le mie sostanze e dessi il mio corpo per esser bruciato, ma non avessi la carità, niente mi giova.
	% La carità è paziente, è benigna la carità; non è invidiosa la carità, non si vanta, non si gonfia, non manca di rispetto, non cerca il suo interesse, non si adira, non tiene conto del male ricevuto, non gode dell'ingiustizia, ma si compiace della verità. Tutto copre, tutto crede, tutto spera, tutto sopporta.
	% La carità non avrà mai fine. Le profezie scompariranno; il dono delle lingue cesserà e la scienza svanirà. La nostra conoscenza è imperfetta e imperfetta la nostra profezia. Ma quando verrà ciò che è perfetto, quello che è imperfetto scomparirà. Quand'ero bambino, parlavo da bambino, pensavo da bambino, ragionavo da bambino. Ma, divenuto uomo, ciò che era da bambino l'ho abbandonato.
	% Ora vediamo come in uno specchio, in maniera confusa; ma allora vedremo a faccia a faccia. Ora conosco in modo imperfetto, ma allora conoscerò perfettamente, come anch'io sono conosciuto. Queste dunque le tre cose che rimangono: la fede, la speranza e la carità; ma di tutte più grande è la carità!
	% Ricercate la carità.}

\subsection*{Canto al Vangelo: \textit{Alleluia}}

	\begin{verse}
		\begin{mystrofe}
			\ritornello{Alleluia, allelu-alleluia
		Alleluia, alleluia
		Alleluia, allelu-alleluia
		Alleluia, alleluia}
		\end{mystrofe}

		\begin{mystrofe}
			Canto per Cristo che mi libererà quando verrà nella gloria, quando la vita con lui rinascerà, alleluia, alleluia!
		\end{mystrofe}

		\begin{mystrofe}
			\ritornello{Alleluia, allelu-alleluia
			Alleluia, alleluia
			Alleluia, allelu-alleluia
			Alleluia, alleluia}
		\end{mystrofe}

	\end{verse}

	\newpage

\subsection*{Vangelo}

	\vangelo{Dal Vangelo secondo Luca}{Lc 14, 25-33}{
		In quel tempo, una folla numerosa andava con Gesù. Egli si voltò e disse loro: "Se uno viene a me e non mi ama più di quanto ami suo padre, la madre, la moglie, i figli, i fratelli, le sorelle e perfino la propria vita, non può essere mio discepolo. Colui che non porta la propria croce e non viene dietro a me, non può essere mio discepolo. Chi di voi, volendo costruire una torre, non siede prima a calcolare la spesa e a vedere se ha i mezzi per portarla a termine? Per evitare che, se getta le fondamenta e non è in grado di finire il lavoro, tutti coloro che vedono comincino a deriderlo, dicendo: "Costui ha iniziato a costruire, ma non è stato capace di finire il lavoro". Oppure quale re, partendo in guerra contro un altro re, non siede prima a esaminare se può affrontare con diecimila uomini chi gli viene incontro con ventimila? Se no, mentre l'altro è ancora lontano, gli manda dei messaggeri per chiedere pace.

		Così chiunque di voi non rinuncia a tutti i suoi averi, non può essere mio discepolo".
	}

\subsection*{Omelia del Sacerdote}

\newpage


% --------------------------------------------------

% Rito del Matrimonio
\section*{Rito del Matrimonio}

	\begin{dialoghi}
		\item[\sacerdote] Carissimi Irene e Davide, siete venuti nella casa del Signore, davanti al ministro della Chiesa e davanti alla comunità, perché la vostra decisione di unirvi in matrimonio riceva il sigillo dello Spirito Santo, sorgente dell'amore fedele e inesauribile. Ora Cristo vi rende partecipi dello stesso amore con cui egli ha amato la sua Chiesa, fino a dare se stesso per lei. Vi chiedo pertanto di esprimere le vostre intenzioni.
	\end{dialoghi}

\subsection*{Domande}

	\begin{dialoghi}
		\item[\sacerdote] Irene e Davide siete venuti a contrarre matrimonio, senza alcuna costrizione, in piena libertà e consapevoli del significato della vostra decisione?
		\item[\sposi] Sì.
		\item[\sacerdote] Siete disposti nella nuova via del matrimonio ad amarvi e onorarvi l'un l'altro per tutta la vita?
		\item[\sposi] Sì.
		\item[\sacerdote] Siete disposti ad accogliere con amore i figli che Dio vorrà donarvi e a educarli secondo la legge di Cristo e della sua Chiesa?
		\item[\sposi] Sì.
	\end{dialoghi}

\subsection*{Consenso}

	\begin{dialoghi}
		\item[\sacerdote] Alla presenza di Dio e davanti alla Chiesa qui riunita, datevi la mano destra ed esprimete il vostro consenso. Il Signore, inizio e compimento del vostro amore, sia con voi sempre.
		\item[\sposo] Io Davide accolgo te, Irene, come mia sposa. Con la grazia di Cristo prometto di esserti fedele sempre, nella gioia e nel dolore, nella salute e nella malattia, e di amarti e onorarti tutti i giorni della mia vita.
		\item[\sposa] Io Irene accolgo te, Davide, come mio sposo. Con la grazia di Cristo prometto di esserti fedele sempre, nella gioia e nel dolore, nella salute e nella malattia, e di amarti e onorarti tutti i giorni della mia vita.
		\item[\sacerdote] Il Dio di Abramo, il Dio di Isacco, il Dio di Giacobbe, il Dio che nel paradiso ha unito Adamo ed Eva confermi in Cristo il consenso che avete manifestato davanti alla Chiesa e vi sostenga con la sua benedizione. L'uomo non osi separare ciò che Dio unisce.
		\item[\assemblea] Amen.
	\end{dialoghi}

\subsection*{Benedizione e consegna degli anelli}

	\begin{dialoghi}
		\item[\sacerdote] Signore, benedici e santifica l'amore di questi sposi: l'anello che porteranno come simbolo di fedeltà li richiami continuamente al vicendevole amore. Per Cristo nostro Signore.
		\item[\assemblea] Amen.
		\item[\sposo] Irene, ricevi questo anello, segno del mio amore e della mia fedeltà. Nel nome del Padre, del Figlio e dello Spirito Santo.
		\item[\sposa] Davide, ricevi questo anello, segno del mio amore e della mia fedeltà. Nel nome del Padre, del Figlio e dello Spirito Santo.
	\end{dialoghi}

\subsection*{Benedizione degli sposi}

	\begin{dialoghi}
		\item[\sacerdote] Fratelli e sorelle, invochiamo con fiducia il Signore, perché effonda la sua grazia e la sua benedizione su questi sposi che celebrano in Cristo il loro matrimonio: egli che li ha uniti nel patto santo [per la comunione al corpo e al sangue di Cristo] li confermi nel reciproco amore.

		O Dio, con la tua onnipotenza hai creato dal nulla tutte le cose e nell'ordine primordiale dell'universo hai formato l'uomo e la donna a tua immagine, donandoli l'uno all'altro come sostegno inseparabile, perché siano non più due, ma una sola carne; così hai insegnato che non è mai lecito separare ciò che tu hai costituito in unità.

		O Dio, in un mistero così grande hai consacrato l'unione degli sposi e hai reso il patto coniugale sacramento di Cristo e della Chiesa.

		O Dio, in te la donna e l'uomo si uniscono, e la prima comunità umana, la famiglia, riceve in dono quella benedizione che nulla poté cancellare, né il peccato originale né le acque del diluvio.

		Guarda ora con bontà questi tuoi figli che, uniti nel vincolo del matrimonio, chiedono l'aiuto della tua benedizione: effondi su di loro la grazia dello Spirito Santo perché, con la forza del tuo amore diffuso nei loro cuori, rimangano fedeli al patto coniugale.

		In questa tua figlia Irene, dimori il dono dell'amore e della pace e sappia imitare le donne sante lodate dalla Scrittura. Davide, suo sposo, viva con lei in piena comunione, la riconosca partecipe dello stesso dono di grazia, la onori come uguale nella dignità, la ami sempre con quell'amore con il quale Cristo ha amato la sua Chiesa

		Ti preghiamo, Signore, affinché questi tuoi figli rimangano uniti nella fede e nell'obbedienza dei tuoi comandamenti; fedeli a un solo amore siano esemplari per integrità di vita; sostenuti dalla forza del Vangelo diano a tutti buona testimonianza di Cristo.

		Sia feconda la loro unione, diventino genitori saggi e forti e insieme possano vedere i figli dei loro figli. E dopo una vita lunga e serena giungano alla beatitudine eterna del regno dei cieli. Per Cristo nostro Signore.
		\item[\assemblea] Amen.
		\item[\sacerdote] Benediciamo il Signore.
		\item[\assemblea] A lui onore e gloria nei secoli.
	\end{dialoghi}



% --------------------------------------------------

\subsection*{PREGHIERA DEI FEDELI}

\begin{dialoghi}
\item[Sacerdote] Fratelli e sorelle, accompagniamo con le nostre preghiere questa nuova famiglia, perché per l'intercessione dei santi, si accresca di giorno in giorno il reciproco amore di questi sposi e Dio sostenga nella sua bontà tutte le famiglie. Ripetiamo insieme: "ti preghiamo, ascoltaci".
\item[Assemblea] Ti preghiamo, ascoltaci
\item[Lettore] Per i nuovi sposi Irene e Davide, perché la loro famiglia cresca nell'unità e nella pace, invochiamo il Signore.
\item[Assemblea] Ti preghiamo, ascoltaci
\item[Lettore] Per i loro parenti e amici e per tutti coloro che sono stati di aiuto a questi sposi, invochiamo il Signore.
\item[Assemblea] Ti preghiamo, ascoltaci
\item[Lettore] Per i giovani che si stanno preparando a celebrare il matrimonio e per tutti coloro che Dio chiama ad altre scelte di vita, invochiamo il Signore.
\item[Assemblea] Ti preghiamo, ascoltaci
\item[Lettore] Per tutte le famiglie e perché fra gli uomini si stabilisca una pace duratura, invochiamo il Signore.
\item[Assemblea] Ti preghiamo, ascoltaci
\item[Lettore] Per i defunti che hanno lasciato questo mondo e in particolare per i nostri familiari e amici, invochiamo il Signore.
\item[Assemblea] Ti preghiamo, ascoltaci
\item[Lettore] Per la Chiesa, popolo santo di Dio, e per l'unità di tutti i cristiani, invochiamo il Signore.
\item[Assemblea] Ti preghiamo, ascoltaci
\item[Sacerdote] Signore Gesù, che sei presente in mezzo a noi, accogli la nostra preghiera mentre Irene e Davide consacrano la loro unione, e riempici del tuo Spirito. Tu che vivi e regni nei secoli dei secoli.
\item[Assemblea] Amen.
\item[Sacerdote] Ora, in comunione con la Chiesa del cielo, invochiamo l'intercessione dei santi

Santa Maria, Madre di Dio, \textbf{prega per noi}\\
Santa Maria, Madre della Chiesa, \textbf{prega per noi}\\
Santa Maria, Regina della Famiglia, \textbf{prega per noi}\\
San Giuseppe, Sposo di Maria, \textbf{prega per noi}\\
Santi Angeli di Dio, \textbf{pregate per noi}\\
Santi Gioacchino e Anna, \textbf{pregate per noi}\\
Santi Zaccaria e Elisabetta, \textbf{pregate per noi}\\
San Giovanni Battista, \textbf{prega per noi}\\
Santi Pietro e Paolo, \textbf{pregate per noi}\\
Santi Apostoli ed Evangelisti, \textbf{pregate per noi}\\
Santi Martiri di Cristo, \textbf{pregate per noi}\\
Santi Aquila e Priscilla, \textbf{pregate per noi}\\
Santi Mario e Marta, \textbf{pregate per noi}\\
Santa Monica, \textbf{prega per noi}\\
San Paolino, \textbf{prega per noi}\\
Santa Brigida, \textbf{prega per noi}\\
Santa Rita, \textbf{prega per noi}\\
Santa Francesca Romana, \textbf{prega per noi}\\
San Tommaso Moro, \textbf{prega per noi}\\
Santa Giovanna Beretta Molla \textbf{prega per noi}\\
San Davide \textbf{prega per noi}\\
Santa Irene \textbf{prega per noi}\\
San Pio X \textbf{prega per noi}\\
Santi e Sante tutte \textbf{pregate per noi}
\item[Sacerdote] Effondi, Signore, su Irene e Davide lo Spirito del tuo amore, perché diventino un cuor solo e un'anima sola: nulla separi questi sposi che tu hai unito, e, ricolmati della tua benedizione, nulla li affligga. Per Cristo nostro Signore.
\item[Assemblea] Amen
\end{dialoghi}


% --------------------------------------------------

% Liturgia Eucaristica
\section*{LITURGIA EUCARISTICA}

% --------------------------------------------------

% Liturgia Eucaristica
\section*{Liturgia eucaristica}

\subsection*{Canto di offertorio: \textit{Benedici}}

	\begin{verse}
		\begin{mystrofe}
		Nebbia e freddo, giorni lunghi e amari \\
		mentre il seme muore. \\
		Poi il prodigio antico e sempre nuovo \\
		del primo filo d'erba \\
		e nel vento dell'estate ondeggiano le spighe: \\
		avremo ancora pane.
		\end{mystrofe}

		\ritornello{Benedici, o Signore, \\
		questa offerta che portiamo a Te. \\
		Facci uno come il pane \\
		che anche oggi hai dato a noi.}

		\begin{mystrofe}
		Nei filari, dopo il lungo inverno fremono le viti. \\
		La rugiada avvolge nel silenzio i primi tralci verdi, \\
		poi i colori dell'autunno coi grappoli maturi: \\
		avremo ancora vino.
		\end{mystrofe}

		\ritornello{Benedici, o Signore, \\
		questa offerta che portiamo a Te. \\
		Facci uno come il vino \\
		che anche oggi hai dato a noi.}
	\end{verse}

\subsection*{Presentazione dei Doni}

	\begin{dialoghi}
	\item[\sacerdote] Benedetto sei tu, Signore, Dio dell'universo: dalla tua bontà abbiamo ricevuto questo pane, frutto della terra e del lavoro dell'uomo; lo presentiamo a te perché diventi per noi cibo di vita eterna.
	\item[\assemblea] Benedetto nei secoli il Signore
	\item[\sacerdote] Benedetto sei tu, Signore, Dio dell'universo: dalla tua bontà abbiamo ricevuto questo vino, frutto della vite e del lavoro dell'uomo; lo presentiamo a te perché diventi per noi bevanda di salvezza.
	\item[\assemblea] Benedetto nei secoli il Signore
	\item[\sacerdote] Pregate, fratelli, perché il mio e vostro sacrificio sia gradito a Dio, Padre onnipotente.
	\item[\assemblea] Il Signore riceva dalle tue mani questo sacrificio a lode e gloria del suo nome, per il bene nostro e di tutta la sua santa Chiesa.
	\end{dialoghi}

\subsection*{Orazione sui Doni}

	\begin{dialoghi}
	\item[\sacerdote] O Dio, Padre di bontà, accogli il pane e il vino, che la tua famiglia ti offre con intima gioia, e custodisci nel tuo amore Irene e Davide che hai unito col sacramento nuziale. Per Cristo nostro Signore.
	\item[\assemblea] Amen
	\end{dialoghi}

\subsection*{Preghiera Eucaristica}

	\begin{dialoghi}
		\item[\sacerdote] Il Signore sia con voi
		\item[\assemblea] E con il tuo spirito
		\item[\sacerdote] In alto i nostri cuori
		\item[\assemblea] Sono rivolti al Signore
		\item[\sacerdote] Rendiamo grazie al Signore, nostro Dio
		\item[\assemblea] È cosa buona e giusta
		\item[\sacerdote] È veramente cosa buona e giusta, nostro dovere e fonte di salvezza, rendere grazie sempre, qui e in ogni luogo, a te, Signore, Padre santo, Dio onnipotente ed eterno. Tu hai dato alla comunità coniugale la dolce legge dell'amore e il vincolo indissolubile della pace perché l'unione casta e feconda degli sposi accresca il numero dei tuoi figli. Con disegno mirabile hai disposto che la nascita di nuove creature allieti l'umana famiglia e la loro rinascita in Cristo edifichi la tua Chiesa. Per questo mistero di salvezza, uniti agli angeli e a tutti i santi, cantiamo insieme l'inno della tua lode.
	\end{dialoghi}

\subsection*{Canto: \textit{Santo (Jesus Christ Superstar)}}

	\begin{verse}
		\begin{mystrofe}
		Santo, santo, santo è il Signore \\
		il Signore Dio dell'Universo. \\
		I cieli e la terra sono pieni della tua gloria \\
		osanna nell'alto dei cieli. \\
		Benedetto sia colui che viene, \\
		che viene nel nome del Signore. \\
		Osanna, osanna nell'alto dei cieli \\
		osanna nell'alto dei cieli.
		\end{mystrofe}
	\end{verse}

	\begin{dialoghi}
		\item[\sacerdote] Veramente santo sei tu, o Padre, fonte di ogni santità. Ti preghiamo: santifica questi doni con la rugiada del tuo Spirito perché diventino per noi il corpo e (+) il sangue del Signore nostro Gesù Cristo.

		Egli, consegnandosi volontariamente alla passione, prese il pane, rese grazie, lo spezzò, lo diede ai suoi discepoli, e disse: "\textbf{PRENDETE, E MANGIATENE TUTTI: QUESTO É IL MIO CORPO OFFERTO IN SACRIFICIO PER VOI.}"

		Allo stesso modo, dopo aver cenato, prese il calice, di nuovo ti rese grazie, lo diede ai suoi discepoli e disse: "\textbf{PRENDETE, E BEVETENE TUTTI: QUESTO É IL CALICE DEL MIO SANGUE PER LA NUOVA ED ETERNA ALLEANZA, VERSATO PER VOI E PER TUTTI IN REMISSIONE DEI PECCATI. FATE QUESTO IN MEMORIA DI ME.}"

		Mistero della fede.
		\item[\assemblea] Annunciamo la tua morte, Signore, proclamiamo la tua risurrezione, nell'attesa della tua venuta.
		\item[\sacerdote] Celebrando il memoriale della morte e risurrezione del tuo Figlio, ti offriamo, Padre, il pane della vita e il calice della salvezza, e ti rendiamo grazie perché ci hai resi degni di stare alla tua presenza a compiere il servizio sacerdotale. Ti preghiamo umilmente: per la comunione al corpo e al sangue di Cristo lo Spirito Santo ci riunisca in un solo corpo.

		Ricordati, Padre, della tua Chiesa diffusa su tutta la terra e qui convocata nel giorno in cui Cristo ha vinto la morte e ci ha resi partecipi della sua vita immortale: rendila perfetta nell'amore in unione con il nostro Papa Leone, il nostro Vescovo Mario, i presbiteri e i diaconi.

		Ricordati dei tuoi figli Irene e Davide, che in Cristo hanno costituito una nuova famiglia, piccola Chiesa e sacramento del tuo amore, perché la grazia di questo giorno si estenda a tutta la loro vita.

		Ricordati anche dei nostri fratelli e sorelle, che si sono addormentati nella speranza della risurrezione e, nella tua misericordia, di tutti i defunti: ammettili a godere la luce del tuo volto.

		Di noi tutti abbi misericordia, donaci di aver parte alla vita eterna, insieme con la beata Maria, Vergine e Madre di Dio, San Giuseppe suo sposo, gli apostoli e tutti i santi, che in ogni tempo ti furono graditi: e in Gesù Cristo tuo Figlio canteremo la tua gloria.

		Per Cristo, con Cristo e in Cristo, a te, Dio, Padre onnipotente, nell'unita dello Spirito Santo, ogni onore e gloria, per tutti i secoli dei secoli.
		\item[\assemblea] Amen.
	\end{dialoghi}

% --------------------------------------------------


\section*{RITI DI COMUNIONE}

	\begin{dialoghi}
		\item[\sacerdote] Obbedienti alla parola del Salvatore e formati al suo divino insegnamento, osiamo dire:
		\item[\assemblea] Padre nostro, che sei nei cieli, sia santificato il tuo nome, venga il tuo regno, sia fatta la tua volontà, come in cielo così in terra. Dacci oggi il nostro pane quotidiano, e rimetti a noi i nostri debiti come anche noi li rimettiamo ai nostri debitori, e non abbandonarci alla tentazione, ma liberaci dal male.
		\item[\sacerdote] Liberaci, o Signore, da tutti i mali, concedi la pace ai nostri giorni; e con l'aiuto della tua misericordia, vivremo sempre liberi dal peccato e sicuri da ogni turbamento, nell'attesa che si compia la beata speranza, e venga il nostro Salvatore Gesù Cristo.
		\item[\assemblea] Tuo è il regno, tua la potenza e la gloria nei secoli.
		\item[\sacerdote] Signore Gesù Cristo, che hai detto ai tuoi apostoli: "Vi lascio la pace, vi do la mia pace", non guardare ai nostri peccati, ma alla fede della tua Chiesa, e donale unità e pace secondo la tua volontà. Tu che vivi e regni nei secoli dei secoli.
		\item[\assemblea] Amen.
		\item[\sacerdote] La pace del Signore sia sempre con voi.
		\item[\assemblea] E con il tuo spirito.
		\item[\sacerdote] In Cristo, che ci ha resi tutti fratelli con la sua croce, scambiatevi un segno di riconciliazione e di pace.

		\item[\assemblea] Agnello di Dio, che togli i peccati del mondo, abbi pietà di noi. Agnello di Dio, che togli i peccati del mondo, abbi pietà di noi. Agnello di Dio, che togli i peccati del mondo, dona a noi la pace.
		\item[\sacerdote] Ecco l'Agnello di Dio, ecco colui che toglie i peccati del mondo. Beati gli invitati alla cena dell'Agnello.
		\item[\assemblea] O Signore, non son degno di partecipare alla tua mensa: ma dì soltanto una parola e io sarò salvato.
	\end{dialoghi}

\subsection*{CANTO DI COMUNIONE: \textit{Il canto dell'amore}}

	\begin{mystrofe}
		Se dovrai attraversare il deserto \\
		Non temere io sarò con te \\
		Se dovrai camminare nel fuoco \\
		La sua fiamma non ti brucerà \\
		Seguirai la mia luce nella notte \\
		Sentirai la mia forza nel cammino \\
		Io sono il tuo Dio, il Signore \\
		Sono io che ti ho fatto e plasmato \\
		Ti ho chiamato per nome \\
		Io da sempre ti ho conosciuto \\
		E ti ho dato il mio amore \\
		Perché tu sei prezioso ai miei occhi \\
		Vali più del più grande dei tesori \\
		Io sarò con te dovunque andrai \\
		Non pensare alle cose di ieri \\
		Cose nuove fioriscono già \\
		Aprirò nel deserto sentieri \\
		Darò acqua nell\'aridità \\
		Perché tu sei prezioso ai miei occhi \\
		Vali più del più grande dei tesori \\
		Io sarò con te dovunque andrai \\
		Dovunque andrai \\
		Perché tu... \\
		Io ti sarò accanto sarò con te \\
		Per tutto il tuo viaggio sarò con te
	\end{mystrofe}

\subsection*{CANTO DI RINGRAZIAMENTO: \textit{Stai con me}}

	\begin{mystrofe}
		Stai con me, proteggimi \\
		Coprimi con le tue ali, o Dio \\
	\end{mystrofe}

	\begin{mystrofe}
		\ritornello{RIT. Quando la tempesta arriverà \\
		Volerò più in alto insieme a te \\
		Nelle avversità sarai con me \\
		Ed io saprò che tu sei il mio Re.}
	\end{mystrofe}

	\begin{mystrofe}
	Il cuore mio riposa in te \\
	Io vivrò in pace e verità \\
	\end{mystrofe}

	\begin{mystrofe}
		\ritornello{RIT. Quando la tempesta arriverà \\
		Volerò più in alto insieme a te \\
		Nelle avversità sarai con me \\
		Ed io saprò che tu sei il mio Re.}
	\end{mystrofe}

\subsection*{ORAZIONE DOPO LA COMUNIONE}

\begin{dialoghi}
	\item[\sacerdote] Preghiamo. O Padre, che ci hai accolti alla tua mensa, concedi a questa nuova famiglia, consacrata dalla tua benedizione, di essere sempre fedele a te e di testimoniare il tuo amore nella comunità dei fratelli. Per Cristo nostro Signore.
	\item[\assemblea] Amen.
\end{dialoghi}


% --------------------------------------------------

\subsection*{LETTURA DEGLI ARTICOLI DEL CODICE CIVILE}

\begin{dialoghi}
\item[\sacerdote] Carissimi Irene e Davide, avete celebrato il sacramento del Matrimonio manifestando il vostro consenso dinanzi a me ed ai testimoni. Oltre la grazia divina e gli effetti stabiliti dai sacri Canoni, il vostro Matrimonio produce anche gli effetti civili secondo le leggi dello Stato. Vi do quindi lettura degli articoli del Codice civile riguardanti i diritti e i doveri dei coniugi che voi siete tenuti a rispettare ed osservare:

\textbf{Art. 143:} Con il matrimonio il marito e la moglie acquistano gli stessi diritti e assumono i medesimi doveri. Dal matrimonio deriva l'obbligo reciproco alla fedeltà, all'assistenza morale e materiale, alla collaborazione nell'interesse della famiglia e alla coabitazione.

Entrambi i coniugi sono tenuti, ciascuno in relazione alle proprie sostanze e alla propria capacità di lavoro professionale o casalingo, a contribuire ai bisogni della famiglia.

\textbf{Art. 144:} I coniugi concordano tra loro l'indirizzo della vita familiare e fissano la residenza della famiglia secondo le esigenze di entrambi e quelle preminenti della famiglia stessa. A ciascuno dei coniugi spetta il potere di attuare l'indirizzo concordato.

\textbf{Art. 147:} «Il matrimonio impone ad ambedue i coniugi l'obbligo di mantenere, istruire, educare e assistere moralmente i figli, nel rispetto delle loro capacità, inclinazioni naturali e aspirazioni, secondo quanto previsto dall'articolo 315-bis».

L'art. 315-bis del codice civile (Diritti e doveri del figlio) così dispone:

«Il figlio ha diritto di essere mantenuto, educato, istruito e assistito moralmente dai genitori, nel rispetto delle sue capacità, delle sue inclinazioni naturali e delle sue aspirazioni. Il figlio ha diritto di crescere in famiglia e di mantenere rapporti significativi con i parenti. Il figlio minore che abbia compiuto gli anni dodici, e anche di età inferiore ove capace di discernimento, ha diritto di essere ascoltato in tutte le questioni e le procedure che lo riguardano. Il figlio deve rispettare i genitori e deve contribuire, in relazione alle proprie capacità, alle proprie sostanze e al proprio reddito, al mantenimento della famiglia finché convive con essa».
\end{dialoghi}

\subsection*{FIRME}

\begin{dialoghi}
\item[\sacerdote] (Spazio per le firme degli sposi e dei testimoni)
\end{dialoghi}


% --------------------------------------------------
\newpage

\section*{Riti di conclusione}

	\begin{dialoghi}
		\item[\sacerdote] Il Signore sia con voi
		\item[\assemblea] E con il tuo spirito.
		\item[\sacerdote] Dio, eterno Padre, vi conservi uniti nel reciproco amore; la pace di Cristo abiti in voi e rimanga sempre nella vostra casa.
		\item[\assemblea] Amen
		\item[\sacerdote] Abbiate benedizione nei figli, conforto dagli amici, vera pace con tutti.
		\item[\assemblea] Amen
		\item[\sacerdote] Siate nel mondo testimoni dell'amore di Dio, perché i poveri e i sofferenti, che hanno sperimentato la vostra carità, vi accolgano grati un giorno nella casa del Padre.
		\item[\assemblea] Amen
		\item[\sacerdote] E su voi tutti che avete partecipato a questa liturgia nuziale, scenda la benedizione di Dio onnipotente, Padre e Figlio (+) e Spirito Santo.
		\item[\assemblea] Amen
		\item[\sacerdote] Nella Chiesa e nel mondo siate testimoni del dono della vita e dell'amore che avete celebrato. Andate in pace
		\item[\assemblea] Rendiamo grazie a Dio
	\end{dialoghi}

\newpage

\subsection*{Canto finale: \textit{Resta accanto a me}}

	\begin{mystrofe}
		\ritornello{RIT. Ora vado sulla mia strada \\
		con l\'amore tuo che mi guida \\
		o Signore, ovunque io vada \\
		resta accanto a me. \\}
	\end{mystrofe}

	\begin{mystrofe}
		Io ti prego, stammi vicino \\
		ogni passo del mio cammino \\
		ogni notte, ogni mattino \\
		resta accanto a me. \\
	\end{mystrofe}

	\begin{mystrofe}
		Il tuo sguardo puro sia luce per me \\
		e la tua parola sia voce per me. \\
		Che io trovi il senso del mio andare \\
		solo in te, \\
		nel tuo fedele amare il mio perché. \\
	\end{mystrofe}

	\begin{mystrofe}
		\ritornello{RIT. Ora vado sulla mia strada \\
		con l\'amore tuo che mi guida \\
		o Signore, ovunque io vada \\
		resta accanto a me. \\}
	\end{mystrofe}

	\begin{mystrofe}
		Fa\' che chi mi guarda non veda che te \\
		fa\' che chi mi ascolta non senta che te \\
		e chi pensa a me, fa\' che nel cuore \\
		pensi a te e trovi quell\'amore \\
		che hai dato a me. \\
	\end{mystrofe}

	\begin{mystrofe}
		\ritornello{RIT. Ora vado sulla mia strada \\
		con l\'amore tuo che mi guida \\
		o Signore, ovunque io vada \\
		resta accanto a me. \\}
	\end{mystrofe}



% --------------------------------------------------
\newpage
\thispagestyle{empty}
\mbox{}   % pagina bianca 1

\newpage
\thispagestyle{empty}
\mbox{}   % pagina bianca 2



\end{document}